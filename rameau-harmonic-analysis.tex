\documentclass{article}
\usepackage{icmc,amsmath}
\usepackage{graphicx}
\usepackage{url}

\title{Rameau: A System for Automatic Harmonic Analysis}

\oneauthor
  {Pedro Kröger, Alexandre Passos, Marcos Sampaio}
  {Genos---Computer Music Research Group\\ School of Music
   \\ Federal University of Bahia, Brazil \\
  \url{pedro.kroger@gmail.com}, \url{alexandre.tp@gmail.com}, \url{mdsmus@gmail.com}}

\newcommand{\nota}[1]{
  \textbf{[nota: \textit{#1}]}
}

\begin{document}
\maketitle

\begin{abstract}
%The abstract should be placed at the top left column and should contain
%about 150-200 words.

\end{abstract}

\section{Introduction}
\label{sec:introduction}

%% introducao geral

In music, harmonic analysis is the study of vertical sonorities and it
conections. The analysit can find the root of chords, label sonorities
with proper chords names (such as ``A minor''), and identify the
relationship beetwen chords, usually using roman numerals.
\nota{nao comeca sentenca com this} This is called functional
analysis. Harmonic analysis is paramount to the understanding of tonal
compositions. The harmonic analysis by computer is an important,
challenging, and interesting computer music research topic. A análise
automática é uma tarefa complexa por diversas razões; o material
musical é composto de uma grande variedade de informações (timbre,
notas, ritmo, dinâmica, harmonia) e pelo fato de música ser um
processo temporal, diferentemente de imagem, por exemplo
\cite{mouton95:numeric}. As mudanças harmônicas em uma textura coral,
onde as notas em todas as vozes começam e terminam ao mesmo tempo, são
óbvias. Contudo, frequentemente os acordes podem ser arpejados,
incompletos, e intercalados com notas não-harmônicas (notas melódicas)
aumentando consideravelmente a complexidade para a análise.
\cite{pardo00:automated}. Also, many techniques such as pattern
matching, hidden Markov model, neural networks, decision trees, etc.
can be used. \nota{voz passiva}

There are many practical applications for automatic music analysis,
among them arranging, detection of possible logical mistakes in scores,
database search, automatic accompaniment generation, and statistical
analysis of musical styles for automated composition
\cite{pardo02:algorithms,temperley99:modeling}.

Computer-based harmonic analysis is important because it can bring new
insights in music theory, in the same way the use of computer in
vision and problem solving has brought new insights in these areas
\cite{temperley99:modeling}.

% porque o trabalho e importante?

Up to now there is not a single framework that allows the comparison
beetwen different algorithms and results. Pardo and Birmmingham state
that ``no researchers have published statistical performance results
on a system designed to analyze tonal music''
\cite{pardo02:algorithms} before their paper. This lack of data in
literature makes it difficult to compare different systems. Also,
there aren't standard benchmarks to compare different algorithms and
results. In fact, only Pardo et al. \cite{pardo00:automated} and
Barthèlemy and Sleator \cite{barthelemy01:figured} published specific
comparisons between theirs and other's results. Pardo compares his
results with \cite{temperley99:modeling} while Barthèlemy compares his
model against \cite{maxwell92:expert}, \cite{pardo99:automated} and
\cite{temperley96:algorithm}. However, they are based on the results
published in papers and not on results from direct implementations,
which means that only the examples published by the authors can be
compared.

In this paper we will present rameau, a system for automatic harmonic
analysis of Western tonal music. The main goal of this systems to
allow an easy comparison beetwen algorithms and results. \nota{falar
  limitacoes e escopo (harmonia tonal, corais de bach)}

In this paper we present an evaluation of five algorithms using a
corpus of 1xx Bach Chorales. \nota{quantos corais?} Section
\ref{sec:system} describes rameau in more details, section
\ref{sec:problems} lists the problems related to harmonic analysis,
section \ref{sec:analysis-results} analysis the results obtained, and
section \ref{sec:concl-future-work} xxxxxxxxx.

\section{Problems}
\label{sec:problems}

* enarmonia

* segmentacao

* analise

* contexto

* notas melodicas

* acordes incompletos

\section{Codificacao jamary}
\label{sec:codificacao-jamary}

\section{Formato dos acordes}
\label{sec:formato-dos-acordes}


\section{The system}
\label{sec:system}

To solve this problem \nota{sim, eu sei, isso não foi escrito ainda,
  mas eu acho que é uma boa ponte} we have designed a framework \nota{
  minha surpresa é que rameau realmente se categoriza como framework,
  se você olhar do ponto de vista de um algoritmo}, Rameau, that
enables us to
\begin{list}{\textbf{.}}{}
\item compare results precisely and repeatably with human analysis;
\item preserve as much information as possible from the score used as
  input;
\item develop new algorithms and test existing ones easilly.
\end{list}

We have, then, started building a corpus of analyzed and digitalized
bach chorales, to use as training and test data. These chorales were
chosen for their ease of analysis (i.e., the segmentation problem in
them reduces to separating the sonorities), their status as canonical
examples of tonal harmony, their number (there are 371 on the
riemenschneider collection) and their simple structure. We analysed
116 of them, and plan on having all properly processed soon. The
chorales are stored in a subset of the GNU Lilypond \nota{citation
  needed} format, from which we generate MIDI files and pictorial
representations of their scores. Our system also generates scores
annotated with analysys results (both by human and by computer).

Internally, the score being analysed is represent as a list of notes,
notated by pitch, octave, onset time and duration. Pitches and
intervals are represented in two ways: standard tempered notation ($C
= 0$, $B = 11$, intervals are the difference between the corresponding
pitches) and a tonal notation developed in \nota{citation needed}. As
of now no implemented algorithm makes use of enharmonic information,
but doing so is trivial.

The results of human analysis performed on the chorales are stored in
a simple, flexible text format, designed to be as close as possible to
usual popular notation, to represent inherent ambiguities in chords
(it is possible, for any sonority, to specify more than one chord as a
possibility for it) and to single out sonorities that do not
constitute a chord, instead serving a melodic purpose on the
piece. This information is then used as test and, when necessary,
training sets on the many algorithms implemented in our system.

Algorithms used for analysis are implemented using a very simple
Common Lisp API. We have currently implemented a subset of the
algorithm described in \cite{pardo02:algorithms} (ignoring, for now,
the segmentation algorithm), a port (work-in-progress) of the
algorithm described in \cite{temperley99:modeling}, and some
neural networks (using the Fann \nota{citation needed} library) and
decision trees (using code from \nota{citar Tom Mitchell, 1997,
  Machine Learning}).

The architecture is simple. First, a piece o music is parsed (our
parser is written using Cl-Yacc and Cl-Lexer \nota{citation needed})
into a list of notes, which is then split into a list of
sonorities. Then, these sonorities are sent to each algorithm for
analysis. The algorithms are expected to return a list of chords or
melodic notes, one for each sonority, which is then compared against
results of human analysis. The comparison and analysis results are
then output in the desired manner.

\nota{screenshots da análise ficariam bem aqui, ou talvez um diagrama
  do processo}

Rameau differs from Temperley and Sleator's Melisma
\cite{temperley99:modeling} system in being focused on testing the
results. To run Melisma one uses a shell script or pipes, and the
default output of the harmonic analysis program is not suitable for
comparison with previous results. Also, we haven't been able to
reproduce their claimed accuracy using the information avaliable on
the website.
 
\nota{trocar titulo}

\section{Analysis results}
\label{sec:analysis-results}

\section{Conclusions and future work}
\label{sec:concl-future-work}

% planos futuros

* implement other techniques and algorithms such as hidden Markov
chains, patter matching, etc.

* aprofundar o uso de redes neurais

* ampliar o corpus musical (com gabaritos)

* implementar segmentacao

* investigar enarmonia: pitch speller vs. entrada/saida tonal

* implementar analise funcional

\bibliographystyle{plain}
\bibliography{rameau,analise-harmonica-nao-tenho}

\end{document}
