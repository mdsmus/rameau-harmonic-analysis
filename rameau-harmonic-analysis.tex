\documentclass{article}
\usepackage{icmc,amsmath}
\usepackage{graphicx}
\usepackage{url}
\usepackage[utf8x]{inputenc}
\usepackage[english]{babel}
\usepackage{color}

\title{Rameau: A System for Automatic Harmonic Analysis}

\oneauthor
  {Pedro Kröger, Alexandre Passos, Marcos Sampaio}
  {Genos---Computer Music Research Group\\ School of Music
   \\ Federal University of Bahia, Brazil \\
  \url{pedro.kroger@gmail.com}, \url{alexandre.tp@gmail.com}, \url{mdsmus@gmail.com}}

\newcounter{notacounter}

\newcommand{\nota}[1]{
  \addtocounter{notacounter}{1}
  \textcolor{red}{[nota \arabic{notacounter}: #1]}
}


\begin{document}
\maketitle

\begin{abstract}
%The abstract should be placed at the top left column and should contain
%about 150-200 words.

\end{abstract}

\section{Introduction}
\label{sec:introduction}

%% introducao geral

In music, harmonic analysis is the study of vertical sonorities and it
conections. The analysit can find the root of chords, label sonorities
with proper chords names (such as ``A minor''), and identify the
relationship beetwen chords, usually using roman numerals.
\nota{nao comeca sentenca com this} This is called functional
analysis. Harmonic analysis is paramount to the understanding of tonal
compositions. The harmonic analysis by computer is an important,
challenging, and interesting computer music research topic. A análise
automática é uma tarefa complexa por diversas razões; o material
musical é composto de uma grande variedade de informações (timbre,
notas, ritmo, dinâmica, harmonia) e pelo fato de música ser um
processo temporal, diferentemente de imagem, por exemplo
\cite{mouton95:numeric}. As mudanças harmônicas em uma textura coral,
onde as notas em todas as vozes começam e terminam ao mesmo tempo, são
óbvias. Contudo, frequentemente os acordes podem ser arpejados,
incompletos, e intercalados com notas não-harmônicas (notas melódicas)
aumentando consideravelmente a complexidade para a análise.
\cite{pardo00:automated}. Also, many techniques such as pattern
matching, Hidden Markov model, neural networks, decision trees, etc.
can be used. \nota{voz passiva}

There are many practical applications for automatic music analysis,
among them arranging, detection of possible logical mistakes in scores,
database search, automatic accompaniment generation, and statistical
analysis of musical styles for automated composition
\cite{pardo02:algorithms,temperley99:modeling}.

Computer-based harmonic analysis is important because it can bring new
insights in music theory, in the same way the use of computer in
vision and problem solving has brought new insights in these areas
\cite{temperley99:modeling}.

% porque o trabalho e importante?

Up to now there is not a single framework that allows the comparison
beetwen different algorithms and results. Pardo and Birmmingham state
that ``no researchers have published statistical performance results
on a system designed to analyze tonal music''
\cite{pardo02:algorithms} before their paper. This lack of data in
literature makes it difficult to compare different systems. Also,
there aren't standard benchmarks to compare different algorithms and
results. In fact, only Pardo et al. \cite{pardo00:automated} and
Barthèlemy and Sleator \cite{barthelemy01:figured} published specific
comparisons between theirs and other's results. Pardo compares his
results with \cite{temperley99:modeling} while Barthèlemy compares his
model against \cite{maxwell92:expert}, \cite{pardo99:automated} and
\cite{temperley96:algorithm}. However, they are based on the results
published in papers and not on results from direct implementations,
which means that only the examples published by the authors can be
compared.

In this paper we will present rameau, a system for automatic harmonic
analysis of Western tonal music. The main goal of this systems to
allow an easy comparison beetwen algorithms and results. \nota{falar
  limitacoes e escopo (harmonia tonal, corais de bach)}

In this paper we present an evaluation of five algorithms using a
corpus of 1xx Bach Chorales. \nota{quantos corais?} Section
\ref{sec:system} describes rameau in more details, section
\ref{sec:problem} lists the problems related to harmonic analysis,
section \ref{sec:analysis-results} analysis the results obtained, and
section \ref{sec:concl-future-work} xxxxxxxxx.

\section{The problem of automated harmonic analysis}
\label{sec:problem}

The problem of automated harmonic analysis might be understood by
dividing it into four sub-problems. 

The first problem has to do with the kind of input to the system. If
the input doesn't store enharmonic information the program may have
problems in places where enharmonic information is important. For
example, in \cite{pardo02:algorithms} a german augmented sixth is
wrongly classified as a dominant sevent chord. A solution for this
problem is or to use a notation that retain enharmonic spelling or
process the input using a pitch speller. Usually systems or ignore
this problem (like the above mentioned citation) or develop a pitch
speller \nota{citation needed}. To our knowledge no research dealing
with harmonic analysis has used a notation that takes enharmonic
differences in consideration as input. \nota{isso é verdade? passos:
  como entrada, até agora sim, mas temperley, por exemplo, diferencia
  enarmonias na saída}

The second problem is to divide a given piece into segments with
harmonic significance. \nota{falar mais sobre segmentação}

The third is to label each segment with a chord name. \nota{falar
  sobre problemas de notas melodicas, acordes incompletos, contexto}

The last problem is to assign a tonal function to each chord.
\nota{expandir nos problemas especificos}

\section{The Rameau Framework}
\label{sec:system}

To properly study, understand and compare existing and newly proposed
solutions for the automated harmonic analysis problem we have designed
and implemented a framework , Rameau, that should enable us to

\begin{enumerate}
\item compare results precisely and repeatably with human analysis of
  a large corpus of music,
\item allow algorithms to access as much information as possible from
  the score used as input, and
\item easily develop new algorithms, test existing ones, and compare
  precisely the results.
\end{enumerate}

\subsection{Comprehensive Test Corpus}
\label{sec:comprehensive-test-corpus}

We are building a corpus of analyzed and digitalized Bach chorales to
use as training and test data. Bach chorales were chosen as our first
corpus for their ease of analysis (i.e., the segmentation problem in
them reduces to separating the sonorities), their status as canonical
examples of tonal harmony, their number (there are 371 on the
Riemenschneider collection) and their simple structure. We have
already analyzed 150 of them and plan on having all properly processed
soon. The chorales are stored in a subset of the GNU Lilypond
\cite{nienhuys:lilypond} format, from which we generate MIDI files and
typeset scores. Our system also generates scores annotated with
analysis results (both by human and by computer).

\subsection{Tonal codification}
\label{sec:codificacao-jamary}

\nota{usar tonal, base-40, e binomial systems?}

Probably the pitch-class notation, or some variation such as MIDI, is
the most used way to represent pitch numerically. The main problem of
this notation is that it is imposible to retaim enharmonic spelling. A
few systems have been developed to address this problem. 

Brinkman's sytem \cite{brinkman86:_binom_repres_of_pitch_for}
represents notes in a format \texttt{<pc,nc>} where pc stands for
pitch-class and nc for note-class. Note-class is a modulo 7 notation
system that represents numerically each note name (A, B, C, D, E, F,
G) from 0 to 6. In this system C$\sharp$ is represented as
\texttt{<1,0>} and D$\flat$ as \texttt{<1,0>}. Brinkman also proposes
a packged representation using single numbers in the formula $bc =
pc\times10 + nc$. For instance, C$\sharp$ = 10 and D$\flat$ =
11.\nota{esses valores são inconsistentes entre si, não?}

Hewlett proposes a base-40 pitch notation that uses only single
numbers \cite{hewlett92:base40}. In this system C$\sharp$ = 4 and
D$\flat$ = 8, for instance.

In rameau we use Oliveira's base-96 system
\cite{oliveira01:codificacao}. Oliveira's system is simple, elegant,
and overcomes a few shortcommings in both Brinkman's and Hewlett's.
Since the original publication of Oliveira's system is available only
in Portuguese, we will describe it here, comparing with the other
systems.

Like Hewlett's, Oliveira's system assign each note with a number. For
instance C = 0, G = 55, and so on up to 96. Table
\ref{tab:jama-notas} shows all notes in the system. In this system the
intervals are invariant under transformation. Table
\ref{tab:jama-intervalos} shows the coding for the intervals.

\begin{table}
  \centering
  \begin{tabular}{l|lllllll}
               & c & d& e& f& g& a& b \\
    \hline
    7$\flat$   &   & 7&21&  &  &62&76 \\
    6$\flat$   & 90& 8&22&35&49&63&77 \\
    5$\flat$   & 91& 9&23&36&50&64&78 \\
    4$\flat$   & 92&10&24&37&51&65&79 \\
    3$\flat$   & 93&11&25&38&52&66&80 \\
    2$\flat$   & 94&12&26&39&53&67&81 \\
    $\flat$    & 95&13&27&40&54&68&82 \\
    $\natural$ &  0&14&28&41&55&69&83 \\
    $\sharp$   &  1&15&29&42&56&70&84 \\
    2$\sharp$  &  2&16&30&43&57&71&85 \\
    3$\sharp$  &  3&17&31&44&58&72&86 \\
    4$\sharp$  &  4&18&32&45&59&73&87 \\
    5$\sharp$  &  5&19&33&46&60&74&88 \\
    6$\sharp$  &  6&20&34&47&61&75&89 \\
    7$\sharp$  &   &  &  &48&  &  &   \\
  \end{tabular}
  \caption{Notes in Oliveira's system}
  \label{tab:jama-notas}
\end{table}

\begin{table}
  \centering
  \begin{tabular}{l|llllllll}
    & 1$^{a}$& 2$^{a}$& 3$^{a}$& 4$^{a}$& 5$^{a}$& 6$^{a}$& 7$^{a}$& 8$^{a}$ \\
    \hline
    sd  &  & 7&21&35&49&62&76&90 \\
    pd  &  & 8&22&36&50&63&77&91 \\
    qd  &  & 9&23&37&51&64&77&92 \\
    td  &  &10&24&38&52&65&78&93 \\
    dd  &  &11&25&39&53&66&79&94 \\
    d   &  &12&26&40&54&67&80&95 \\
    m   &  &13&27&  &  &68&82&   \\
    J   & 0&  &  &41&55&  &  &96 \\
    M   &  &14&28&  &  &69&83&   \\
    A   & 1&15&29&42&56&70&84&   \\
    dA  & 2&16&30&43&57&71&85&   \\
    tA  & 3&17&31&44&58&72&86&   \\
    qA  & 4&18&32&45&59&73&87&   \\
    pA  & 5&19&33&46&60&74&88&   \\
    sA  & 6&20&34&47&61&75&89&   \\
    hA  &  &  &  &48&  &  &  &   
  \end{tabular}
  \caption{Invervals in Oliveira's system}
  \label{tab:jama-intervalos}
\end{table}

Oliveira's system works up to seven flats and sharps, witch is more
than enough for any aplication of tonal music. Because this system is
modulo 96, it is compatible with the pitch-class (modulo 12) system.
An operation of modulo 12 in any note in oliveira's system will
convert it to the pitch-class. For example, in oliveira's system G =
55. The result of 55 mod 12 is 7.

Hewlett's system starts to fail with notes that have more than two
accidentals. For example, the interval beetween C$\flat\flat$ and
C$\sharp\sharp$ (a quadruple augmented unison) has the same value (6)
as diminished second. On the other hand, in Oliveira's system this
interval can be correctely computated.

The main problem with the binomial system is that operations that are
simple become complex. In both tonal and base-40 systems transposition
is as simple as A + B, while in the binomial system it has two
different operations (mod and div). It's packaged representation is
intended to simplify things, but as an example, transposition is
something like:

\begin{verbatim}
10 ((a div 10 + B div 10) mod 12)
 + ((a mod 10 + B div 10) mod 7)
\end{verbatim}

To do anything more complex, like interval-class vectors, yet extra
work has to be done, increasing complexity.

For these reasons we have adopted Oliveira's codification, although no
algorithm at present makes use of it.

\subsection{Answer sheet format}
\label{sec:formato-dos-acordes}

The results of human analysis performed on the chorales are stored in
a simple, flexible text format, designed to

\begin{enumerate}
\item be as close as possible to usual popular notation,
\item represent inherent ambiguities in chords (it is possible to name
  more than one chord as a possible analysis for a given sonority),
  and
\item single out sonorities that do not constitute chords, instead
  having melodic functions on the piece.
\end{enumerate}

The beginning of the answer sheet for Bach's ``Aus meines Herzens
Grunde'' chorale (the first in our corpus), for example, looks like
\texttt{G G C/E (C7+/E [b])}. Each chord represents a sonority. Chords
in parenthesis represent possible interpretations of a single
sonority. Notes in brackets are non-chord tones, marking sonorities
that do not constitute chords.

This information is then used as test and training sets on the many
algorithms implemented in our system.

\subsection{Architecture}
\label{sec:architecture-and-api}

The architecture of the Rameau framework is simple. First, a score in
a GNU Lilypond-compatible format is parsed (our parser is written
using Cl-Yacc \cite{chroboczek:_cl_yacc_manual} and Cl-Lexer
\cite{parker:_lexer_packag}) into a list of notes, which is then split
into a list of sonorities. Then, these sonorities are sent to each
algorithm for analysis. The analysis results and their comparison with
the answer sheets are then output either textually or as an annotated
score.

The algorithms used for analysis are implemented using a very simple
Common Lisp API. To implement one it is enough to place a lisp file in
a special directory and call the \texttt{registra-algoritmo} function
inside that file, specifying as parameter the algorithm's name, an
analysis function and a comparison function. The input to the analysis
function is a list of sonorities, which are lists of notes, each note
specifying its onset time, duration, octave and pitch. The output is a
list of either chords or non-chord tones, each chord possibly having a
fundamental, a mode, a seventh, a bass, and additions. The framework
provides a few possible comparison functions for usage.

We have currently implemented using this API 
\begin{enumerate}
\item a subset of the algorithm described in \cite{pardo02:algorithms}
  (ignoring, for now, segmentation), 
\item a port (work-in-progress) of the algorithm described in
  \cite{temperley99:modeling}, 
\item some neural networks (using the Fann \nota{citation needed}
  library) roughly similiar to some described in
  \cite{tsui02:_harmon_analy_using_neural_networ}, and
\item decision trees modeled after the neural networks (using code
  from \cite{Mitchell:1997:ML}).
\end{enumerate}

The code is publicly available, under a GNU GPL \nota{citation needed}
license in a git \cite{baudis:_git_users_manual} repository
at:\\ \texttt{http://genos.mus.br/repos/rameau.git}\\ and for
visualization at\\ \texttt{http://git.genos.mus.br/?p=rameau.git}

\subsection{Comparison with similar software}
\label{sec:differences-from-similar-software}

Rameau differs from Temperley and Sleator's Melisma
\cite{temperley99:modeling} system mainly by our focus on testing the
results of many algorithms. To run Melisma one uses a shell script or
pipes, producing by default a pretty - although hard to parse -
output. It is not trivial to transform this output into a form
suitable for comparison with a human-made answer sheet. For this
reason, also, since our re-implementation of their algorithm is still
not perfect, we haven't been able to reproduce their claimed accuracy
using the information available on their website.

Even though we couldn't find a public repository for Pardo and
Birmingham's HarmAn \cite{pardo99:automated}, from their article the
system seems to face most issues we found with the Melisma system. Our
re-implementation of their algorithm is based solely on the article.

Common to most other systems is the notion that in a musical piece
every sonority is part of a chord or at least has a root. This,
however, is not true of sonorities containing non-chord tones,
suspensions, passing notes and some voice leading clues. It is one of
our goals to get results as close as possible to usual human analysis,
so we classify these sonorities as non-chord tone and consider any
attempt at labeling them as a chord or finding their root an error.
 
\section{Preliminary results}
\label{sec:analysis-results}

\subsection{Training set}

The effect of the training set on the algorithms can be seen in

\begin{figure}
  \includegraphics[scale=0.35,angle=270]{treinamento-corais.eps}
  \caption{Accuracy versus amount of training data per algorithm}
  \label{fig:treinamento-corais}
\end{figure}

\subsection{Hidden Units}

The neural network's performance varies with the number of hidden
units, roughly following a sigmoid curve.

\begin{figure}
  \includegraphics[scale=0.35,angle=270]{dados-simple-net.eps}
  \caption{Accuracy versus number of hidden nodes in the simple-net}
  \label{fig:simple-net}
\end{figure}

\begin{figure}
  \includegraphics[scale=0.35,angle=270]{dados-context-net.eps}
  \caption{Accuracy versus number of hidden nodes in the context-net}
  \label{fig:context-net}
\end{figure}

\begin{figure}
  \includegraphics[scale=0.35,angle=270]{dados-chord-net.eps}
  \caption{Accuracy versus number of hidden nodes in the chord-net}
  \label{fig:chord-net}
\end{figure}

\begin{figure}
  \includegraphics[scale=0.35,angle=270]{dados-mode-net.eps}
  \caption{Accuracy versus number of hidden nodes in the mode-net}
  \label{fig:mode-net}
\end{figure}


\section{Conclusions and future work}
\label{sec:concl-future-work}

% planos futuros

* implement other techniques and algorithms such as Hidden Markov
Model, patter matching, etc.

* aprofundar o uso de redes neurais

* ampliar o corpus musical (com gabaritos)

* implementar segmentacao

* investigar enarmonia: pitch speller vs. entrada/saida tonal

* implementar analise funcional

\bibliographystyle{plain}
\bibliography{rameau,analise-harmonica-nao-tenho,bib-outras,bib-geral}

\end{document}
